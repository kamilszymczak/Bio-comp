%% Generated by Sphinx.
\def\sphinxdocclass{report}
\documentclass[letterpaper,10pt,english]{sphinxmanual}
\ifdefined\pdfpxdimen
   \let\sphinxpxdimen\pdfpxdimen\else\newdimen\sphinxpxdimen
\fi \sphinxpxdimen=.75bp\relax

\PassOptionsToPackage{warn}{textcomp}
\usepackage[utf8]{inputenc}
\ifdefined\DeclareUnicodeCharacter
% support both utf8 and utf8x syntaxes
  \ifdefined\DeclareUnicodeCharacterAsOptional
    \def\sphinxDUC#1{\DeclareUnicodeCharacter{"#1}}
  \else
    \let\sphinxDUC\DeclareUnicodeCharacter
  \fi
  \sphinxDUC{00A0}{\nobreakspace}
  \sphinxDUC{2500}{\sphinxunichar{2500}}
  \sphinxDUC{2502}{\sphinxunichar{2502}}
  \sphinxDUC{2514}{\sphinxunichar{2514}}
  \sphinxDUC{251C}{\sphinxunichar{251C}}
  \sphinxDUC{2572}{\textbackslash}
\fi
\usepackage{cmap}
\usepackage[T1]{fontenc}
\usepackage{amsmath,amssymb,amstext}
\usepackage{babel}



\usepackage{times}
\expandafter\ifx\csname T@LGR\endcsname\relax
\else
% LGR was declared as font encoding
  \substitutefont{LGR}{\rmdefault}{cmr}
  \substitutefont{LGR}{\sfdefault}{cmss}
  \substitutefont{LGR}{\ttdefault}{cmtt}
\fi
\expandafter\ifx\csname T@X2\endcsname\relax
  \expandafter\ifx\csname T@T2A\endcsname\relax
  \else
  % T2A was declared as font encoding
    \substitutefont{T2A}{\rmdefault}{cmr}
    \substitutefont{T2A}{\sfdefault}{cmss}
    \substitutefont{T2A}{\ttdefault}{cmtt}
  \fi
\else
% X2 was declared as font encoding
  \substitutefont{X2}{\rmdefault}{cmr}
  \substitutefont{X2}{\sfdefault}{cmss}
  \substitutefont{X2}{\ttdefault}{cmtt}
\fi


\usepackage[Bjarne]{fncychap}
\usepackage{sphinx}

\fvset{fontsize=\small}
\usepackage{geometry}


% Include hyperref last.
\usepackage{hyperref}
% Fix anchor placement for figures with captions.
\usepackage{hypcap}% it must be loaded after hyperref.
% Set up styles of URL: it should be placed after hyperref.
\urlstyle{same}

\addto\captionsenglish{\renewcommand{\contentsname}{Contents:}}

\usepackage{sphinxmessages}
\setcounter{tocdepth}{1}



\title{Biologically Inspired Computation Coursework}
\date{Oct 09, 2020}
\release{1.0}
\author{Sam Fay\sphinxhyphen{}Hunt, Kamil Szymczak}
\newcommand{\sphinxlogo}{\vbox{}}
\renewcommand{\releasename}{Release}
\makeindex
\begin{document}

\pagestyle{empty}
\sphinxmaketitle
\pagestyle{plain}
\sphinxtableofcontents
\pagestyle{normal}
\phantomsection\label{\detokenize{index::doc}}

\index{module@\spxentry{module}!Coursework.model@\spxentry{Coursework.model}}\index{Coursework.model@\spxentry{Coursework.model}!module@\spxentry{module}}\index{ANN (class in Coursework.model)@\spxentry{ANN}\spxextra{class in Coursework.model}}

\begin{fulllineitems}
\phantomsection\label{\detokenize{index:Coursework.model.ANN}}\pysigline{\sphinxbfcode{\sphinxupquote{class }}\sphinxcode{\sphinxupquote{Coursework.model.}}\sphinxbfcode{\sphinxupquote{ANN}}}
Artificial Neural Network class Implementation
\index{add() (Coursework.model.ANN method)@\spxentry{add()}\spxextra{Coursework.model.ANN method}}

\begin{fulllineitems}
\phantomsection\label{\detokenize{index:Coursework.model.ANN.add}}\pysiglinewithargsret{\sphinxbfcode{\sphinxupquote{add}}}{\emph{\DUrole{n}{layer}}}{}
Add a new layer to the Neural Network
\begin{quote}\begin{description}
\item[{Parameters}] \leavevmode
\sphinxstyleliteralstrong{\sphinxupquote{layer}} ({\hyperref[\detokenize{index:Coursework.model.Layer}]{\sphinxcrossref{\sphinxstyleliteralemphasis{\sphinxupquote{Layer}}}}}) \textendash{} Add an instance of the layer class

\end{description}\end{quote}

\end{fulllineitems}

\index{compile() (Coursework.model.ANN method)@\spxentry{compile()}\spxextra{Coursework.model.ANN method}}

\begin{fulllineitems}
\phantomsection\label{\detokenize{index:Coursework.model.ANN.compile}}\pysiglinewithargsret{\sphinxbfcode{\sphinxupquote{compile}}}{}{}
Compile the Neural Network so it is ready for training or inference

\end{fulllineitems}

\index{epoch() (Coursework.model.ANN method)@\spxentry{epoch()}\spxextra{Coursework.model.ANN method}}

\begin{fulllineitems}
\phantomsection\label{\detokenize{index:Coursework.model.ANN.epoch}}\pysiglinewithargsret{\sphinxbfcode{\sphinxupquote{epoch}}}{\emph{\DUrole{n}{input\_matrix}}}{}
One epoch of the Neural Network
\begin{quote}\begin{description}
\item[{Parameters}] \leavevmode
\sphinxstyleliteralstrong{\sphinxupquote{input\_matrix}} (\sphinxstyleliteralemphasis{\sphinxupquote{numpy.ndarray}}) \textendash{} The matrix provided to the neural network

\item[{Raises}] \leavevmode\begin{itemize}
\item {} 
\sphinxstyleliteralstrong{\sphinxupquote{Exception}} \textendash{} Not Compiled exception

\item {} 
\sphinxstyleliteralstrong{\sphinxupquote{Exception}} \textendash{} Misconfigured input columns exception

\end{itemize}

\end{description}\end{quote}

\end{fulllineitems}

\index{set\_input\_cols() (Coursework.model.ANN method)@\spxentry{set\_input\_cols()}\spxextra{Coursework.model.ANN method}}

\begin{fulllineitems}
\phantomsection\label{\detokenize{index:Coursework.model.ANN.set_input_cols}}\pysiglinewithargsret{\sphinxbfcode{\sphinxupquote{set\_input\_cols}}}{\emph{\DUrole{n}{col\_int}}}{}
Set the number of columns in the input matrix for the ANN
\begin{quote}\begin{description}
\item[{Parameters}] \leavevmode
\sphinxstyleliteralstrong{\sphinxupquote{col\_int}} (\sphinxstyleliteralemphasis{\sphinxupquote{int}}) \textendash{} Integer of the number of rows

\end{description}\end{quote}

\end{fulllineitems}


\end{fulllineitems}

\index{ActivationFunction (class in Coursework.model)@\spxentry{ActivationFunction}\spxextra{class in Coursework.model}}

\begin{fulllineitems}
\phantomsection\label{\detokenize{index:Coursework.model.ActivationFunction}}\pysiglinewithargsret{\sphinxbfcode{\sphinxupquote{class }}\sphinxcode{\sphinxupquote{Coursework.model.}}\sphinxbfcode{\sphinxupquote{ActivationFunction}}}{\emph{\DUrole{n}{value}}}{}
An enumeration.

\end{fulllineitems}

\index{Layer (class in Coursework.model)@\spxentry{Layer}\spxextra{class in Coursework.model}}

\begin{fulllineitems}
\phantomsection\label{\detokenize{index:Coursework.model.Layer}}\pysiglinewithargsret{\sphinxbfcode{\sphinxupquote{class }}\sphinxcode{\sphinxupquote{Coursework.model.}}\sphinxbfcode{\sphinxupquote{Layer}}}{\emph{\DUrole{n}{neurons}}, \emph{\DUrole{n}{activation}\DUrole{o}{=}\DUrole{default_value}{\textquotesingle{}null\textquotesingle{}}}, \emph{\DUrole{n}{use\_bias}\DUrole{o}{=}\DUrole{default_value}{True}}}{}
Layer class used to add layers to the ANN

\end{fulllineitems}

\phantomsection\label{\detokenize{index:module-Coursework.data}}\index{module@\spxentry{module}!Coursework.data@\spxentry{Coursework.data}}\index{Coursework.data@\spxentry{Coursework.data}!module@\spxentry{module}}\index{Data (class in Coursework.data)@\spxentry{Data}\spxextra{class in Coursework.data}}

\begin{fulllineitems}
\phantomsection\label{\detokenize{index:Coursework.data.Data}}\pysiglinewithargsret{\sphinxbfcode{\sphinxupquote{class }}\sphinxcode{\sphinxupquote{Coursework.data.}}\sphinxbfcode{\sphinxupquote{Data}}}{\emph{\DUrole{n}{infile}}, \emph{\DUrole{n}{normalize}\DUrole{o}{=}\DUrole{default_value}{False}}, \emph{\DUrole{n}{delim}\DUrole{o}{=}\DUrole{default_value}{\textquotesingle{}\textbackslash{}t\textquotesingle{}}}}{}
Data class to handle normalization and structure of input data
\index{get\_output() (Coursework.data.Data method)@\spxentry{get\_output()}\spxextra{Coursework.data.Data method}}

\begin{fulllineitems}
\phantomsection\label{\detokenize{index:Coursework.data.Data.get_output}}\pysiglinewithargsret{\sphinxbfcode{\sphinxupquote{get\_output}}}{}{}
Desired result vector
\begin{quote}\begin{description}
\item[{Returns}] \leavevmode
numpy ndarray with the actual results

\item[{Return type}] \leavevmode
ndarray

\end{description}\end{quote}

\end{fulllineitems}

\index{get\_rows() (Coursework.data.Data method)@\spxentry{get\_rows()}\spxextra{Coursework.data.Data method}}

\begin{fulllineitems}
\phantomsection\label{\detokenize{index:Coursework.data.Data.get_rows}}\pysiglinewithargsret{\sphinxbfcode{\sphinxupquote{get\_rows}}}{}{}
Get all the rows of the datafile, excluding the outcome column
\begin{quote}\begin{description}
\item[{Returns}] \leavevmode
returns numpy ndarray of input data

\item[{Return type}] \leavevmode
ndarray

\end{description}\end{quote}

\end{fulllineitems}


\end{fulllineitems}



\chapter{Indices and tables}
\label{\detokenize{index:indices-and-tables}}\begin{itemize}
\item {} 
\DUrole{xref,std,std-ref}{genindex}

\item {} 
\DUrole{xref,std,std-ref}{modindex}

\item {} 
\DUrole{xref,std,std-ref}{search}

\end{itemize}


\renewcommand{\indexname}{Python Module Index}
\begin{sphinxtheindex}
\let\bigletter\sphinxstyleindexlettergroup
\bigletter{c}
\item\relax\sphinxstyleindexentry{Coursework.data}\sphinxstyleindexpageref{index:\detokenize{module-Coursework.data}}
\item\relax\sphinxstyleindexentry{Coursework.model}\sphinxstyleindexpageref{index:\detokenize{module-Coursework.model}}
\end{sphinxtheindex}

\renewcommand{\indexname}{Index}
\printindex
\end{document}